\documentclass[twoside]{article}
\usepackage{multicol}
\setlength{\columnsep}{4cm}
\setlength{\columnseprule}{0pt}
\usepackage[shortlabels]{enumitem}
\setlist[enumerate]{topsep=0pt,itemsep=-0.5ex,partopsep=0.5ex,parsep=0.5ex}

% fonts
    \usepackage[sc]{mathpazo} % Use the Palatino font
    \usepackage[T1]{fontenc} % Use 8-bit encoding that has 256 glyphs
    \linespread{1.05} % Line spacing - Palatino needs more space between lines
    \usepackage{microtype} % Slightly tweak font spacing for aesthetics

    \usepackage[english]{babel} % Language hyphenation and typographical rules

    \usepackage[hmarginratio=1:1,top=32mm,columnsep=20pt]{geometry} % Document margins
    \usepackage[hang, small,labelfont=bf,up,textfont=it,up]{caption} % Custom captions under/above floats in tables or figures
    \usepackage{booktabs} % Horizontal rules in tables

    \usepackage{titling} % Customizing the title section
    \usepackage{abstract} % Allows abstract customization
    \renewcommand{\abstractnamefont}{\normalfont\bfseries} % set abstract title to bold
    \renewcommand{\abstracttextfont}{\normalfont\small\itshape} % set abstract text to small italic
    \renewcommand{\maketitlehookd}{
        \begin{abstract}
        \abstracttext
        \end{abstract}
    }

    \usepackage{titlesec} % Allows customization of titles
    \setlength{\droptitle}{-4\baselineskip} % Move the title up
    \renewcommand\thesection{\Roman{section}} % Roman numerals for the sections
    \renewcommand\thesubsection{\roman{subsection}} % roman numerals for subsections
    \titleformat{\section}[block]{\large\scshape\centering}{\thesection.}{1em}{} % Change the look of the section titles
    \titleformat{\subsection}[block]{\large}{\thesubsection.}{1em}{} % Change the look of the section titles
    \pretitle{\begin{center}\Huge\bfseries} % Article title formatting
    \posttitle{\end{center}} % Article title closing formatting

    \usepackage{fancyhdr} % Headers and footers
    \pagestyle{fancy} % All pages have headers and footers
    \fancyhead{} % Blank out the default header
    \fancyfoot{} % Blank out the default footer
    \fancyhead[C]{} % Custom header text
    \fancyfoot[RO,LE]{\thepage} % Custom footer text

    %\usepackage{scalerel,stackengine}
    %\stackMath
    \usepackage[utf8]{inputenc}
    \usepackage{amsmath} % aligned
    \usepackage{tikz} % overdash
    %\usepackage{MnSymbol} % https://texdoc.org/serve/symbols-a4.pdf/0
    \usepackage{hyperref} % hyperlinks
    \usepackage{mdframed} % boxes
    \hypersetup{
        colorlinks=true,
        linkcolor=cyan,
        filecolor=blue, % magenta
        urlcolor=blue,
        pdftitle={A summary of Projective Geometric Algebra},
        pdfpagemode=FullScreen,
    }
    \makeatletter % allow @ symbol as a character

%\setlength{\skip\footins}{10mm}
%\setlength{\parskip}{1ex}

% utility functions
    \newcommand{\M}[1]{\scriptstyle{#1}} % math text
    \newcommand{\T}[1]{\textrm{#1}} % text in math mode
    \newcommand{\B}[1]{\left(#1\right)} % big brackets
    \newcommand{\F}[2]{\frac{#1}{#2}} % fraction
    \newcommand{\percent}{\T{ \% }}
    \newcommand{\pow}{\T{ }\M{**}\T{ }}
    \newcommand{\rem}{\T{ rem }}
    \newcommand{\ifelse}[3]{#1\T{ ? }#2\T{ : }#3}
    \newcommand{\plusminus}{\pm}
    \newcommand{\minusplus}{\mp}
    \newcommand{\booland}{\T{ }\M{\&\&}\T{ }}
    \newcommand{\boolor}{\T{ }\M{||}\T{ }}
    \newcommand{\bitand}{\T{ }\M{\&}\T{ }}
    \newcommand{\bitor}{\T{ }\M{|}\T{ }}
    \newcommand{\bitxor}{\T{ }\raisebox{0.5ex}{$\scriptstyle\wedge$}\T{ }}
    \newcommand{\rightshift}{\T{ }\M{>>}\T{ }}
    \newcommand{\leftshift}{\T{ }\M{<<}\T{ }}
    \newcommand{\join}{{\,\M{join}\,}}
    \newcommand{\meet}{{\,\M{meet}\,}}
    \newcommand{\anticdot}{~\raisebox{1pt}{\tikz \draw[line width=0.6pt] circle(1.1pt);}~}
    \newcommand{\braket}[1]{\left\langle{#1}\right\rangle}
    \newcommand{\floor}[1]{{\left\lfloor{#1}\right\rfloor}}
    \newcommand{\norm}[1]{{\left\lVert{#1}\right\rVert}}
    \newcommand{\aside}[1]{\begin{flushright}\scriptsize{#1}\end{flushright}}
    \newcommand{\boxedmath}[1]{\begin{center}\boxed{$$#1$$}\end{center}
    }

    % dashed overline
    \newcommand{\overdash}[1]{%
        \tikz[baseline=(todotted.base)]{
            \node[inner sep=1pt,outer sep=0pt] (todotted) {$#1$};
            \draw[dashed, line width=0.05mm] (todotted.north west) -- (todotted.north east);
        }%
    }
    \newcommand{\underdash}[1]{%
        \tikz[baseline=(todotted.base)]{
            \node[inner sep=1pt,outer sep=0pt] (todotted) {$#1$};
            \draw[dashed, line width=0.2mm] (todotted.south west) -- (todotted.south east);
        }%
    }

% header
\title{A summary of Projective Geometric Algebra}
\author{\textsc{Patrolin}}
\date{\today}
\newcommand{\abstracttext}{
    Information on Geometric Algebra is scattered and scarce.
    This is an explanation of how to calculate PGA, but not necessarily why it works
    - for that you can try \href{https://www.youtube.com/playlist?list=PLVuwZXwFua-0Ks3rRS4tIkswgUmDLqqRy}{sudgylacmoe} on youtube and
    \href{https://projectivegeometricalgebra.net/}{Eric Lengyel}'s cheatsheet for point-based PGA (though he goes mad with power).
    We would also point you to \href{https://bivector.net/tools.html}{bivector.net}, but their dual calculations are whack,
    the \href{https://enki.ws/ganja.js/examples/coffeeshop.html\#sUwbwu9vR}{ganja.js} ones are presumably correct, since their demos seem to work.
}

% body
\begin{document}
    \maketitle
    \thispagestyle{empty}

    \begin{multicols*}{2}
        \section{Geometric numbers}
            \centerline{$ 1 + x^2 = 0 $}
            \centerline{$ x = \T{?} $}\vspace{5px}
            \par
                x is not a real number; but if it's not real, why should the other numbers be real?
                \begin{gather*}
                    \B{e_1}^2 + \B{e_2}^2 = \B{e_0}^2 \\
                    \B{e_1}^2 = 1;\, \B{e_2}^2 = -1;\, \B{e_0}^2 = 0
                \end{gather*}
            \par
                In fact, we can define as many of these as we want with a geometric algebra $ \mathbb{G}_{a,b,c} $
                having $ a $ numbers that square to $ 1 $, $ b $ that square to $ -1 $
                and $ c $ that square to $ 0 $ (degenerate basis vectors):
                $$\begin{aligned}
                    &(a + b e_1) \in \mathbb{G}_{1,0,0} \texttt{ // hyperbolic numbers} \\
                    &(a + b e_2) \in \mathbb{G}_{0,1,0} \texttt{ // complex numbers} \\
                    &(a + b e_0) \in \mathbb{G}_{0,0,1} \texttt{ // dual numbers}
                \end{aligned}$$
            \par
                We can multiply these numbers together using the geometric product:
                \begin{gather*}
                    \B{e_i}^2 = \{1, -1, 0\} \\
                    e_i e_j = -e_j e_i
                \end{gather*}
                \aside{The result is another number in the same algebra $ \mathbb{G}_{a,b,c} $}
            \par
                This product is neither commutative nor anticommutative, but it is distributive and associative:
                $$\begin{aligned}
                    AB &\neq BA \\
                    AB &\neq -BA \\
                    A(B+C) &= AB + AC \\
                    (AB)C &= A(BC) \\
                    aB &= Ba;\, a \in \mathbb{R}
                \end{aligned}$$
            \par
                Thus the product of two complex numbers:
                $$\begin{aligned}
                    & (A_1 + A_2 e_2)(B_1 + B_2 e_2) \\
                    & = A_1 B_1 + A_1 B_2 e_2 + A_2 e_2 B_1 + A_2 e_2 B_2 e_2 \\
                    & = A_1 B_1 + A_1 B_2 e_2 + A_2 B_1 e_2 + A_2 B_2 \\
                    & = (A_1 B_1 + A_2 B_2) + (A_1 B_2 + A_2 B_1) e_2
                \end{aligned}$$
        \section{Rotations}
            \par
                A multivector with n basis vectors consists of $ 2^n $ blades:
                \begin{enumerate}[-]
                    \item \texttt{scalar = 0-vector =} $\mathbf{1}$
                    \item \texttt{vector = 1-vector}
                    \item \texttt{bivector = 2-vector}
                    \item \texttt{trivector = 3-vector}
                    \item \texttt{...}
                    \item \texttt{pseudovector = (n-1)-vector}
                    \item \texttt{pseudoscalar = n-vector =} $\mathbb{1}$
                \end{enumerate} \vspace{5px}
                Where a k-vector has $n \choose k$ blades, for example:
                $$\begin{aligned}
                    A = & \, A_1 \\
                    &+ A_2 e_0 + A_3 e_1 + A_4 e_2 \\
                    &+ A_5 e_{01} + A_6 e_{02} + A_7 e_{12} \\
                    &+ A_8 e_{012} \\
                \end{aligned}$$
                \aside{We can abbreviate blades like $ e_1 e_2 $ as $ e_{12} $.}
            \par
                We can use the exponential function to find a rotation $e^A$:
                $$ e^x = \sum_{k=0}^\infty \F{x^k}{k!} $$
                $$\begin{aligned}
                    &e^{a e_2} = 1 + a e_2 - \F{a^2}{2} - \F{a^3}{3} e_2 + ... \\
                    &= (1 - \F{a^2}{2} + ...) + (a - \F{a^3}{3} + ...) e_2 \\
                    &= cos(a) + sin(a) e_2 \\
                \end{aligned}$$
            \par
                Similarly you can find
                $$ e^{e_i} = \begin{cases}
                    cosh(a) + sinh(a) e_i & (\B{e_i}^2 = 1) \\
                    cos(a) + sin(a) e_i & (\B{e_i}^2 = -1) \\
                    1 + e_i & (\B{e_i}^2 = 0)
                \end{cases}$$
            \par
                This gives us hyperbolic rotations, rotations and translations (rotations through infinity) respectively.
            \par
                For the i-th blade $ X_i $ in a multivector:
                \begin{gather*}
                    e^A = \prod_i e^{A_i X_i} \\
                \end{gather*}
        \section{Unary operators}
            \par
                We can define some operations, like reversing the order of
                basis vectors in the blade, that amount to flipping some signs:
                \begin{gather*}
                    \begin{aligned}
                        \tilde{X_i} &= (-1)^\floor{k/2} X_i \texttt{ // reverse} \\
                    \end{aligned} \\
                    X_i \in \T{k-vector} \\
                    f(A) = \sum_i f(X_i)
                \end{gather*}
            \par
                Poincaré duality states that maps between k-vectors and (n-k)-vectors exist.
                \begin{gather*}
                    X_i \, dual(X_i) = \plusminus \mathbb{1} \\
                    dual(X_i) = \plusminus X_{2^n-i+1}
                \end{gather*}
            \par
                For example:
                \begin{gather*}
                    \underline{X_i} X_i = \mathbb{1} \texttt{ // left complement} \\
                    X_i \overline{X_i} = \mathbb{1} \texttt{ // right complement} \\
                    X_i X_i^\star = sign(X^{ND}_i \widetilde{X^{ND}_i}) \, \mathbb{1} \texttt{ // hodge dual}
                \end{gather*}
                Where $ X^{ND}_i $ is $ X_i $ without degenerate basis vectors, e.g.
                $$ X_i = e_{012}; \, X^{ND}_i = e_{12} $$
                \aside{For $ \mathbb{G}_{a,0,c} $: $ X_i^\star = \overline{X_i} $}
            \par
                And applying a dual twice changes the signs,
                so we also want their inverses:
                \begin{gather*}
                    (X_i^\star)^{\star^{-1}} = X_i \\
                    \underline{\overline{X_i}} = X_i
                \end{gather*}
            \par
                In 2D and 3D PGA, we can simplify implementation
                by swapping two basis vectors in some blades such that
                $ \overline{X_i} $ does not flip signs, e.g.
                $$\begin{aligned}
                    A = & \, A_1 + A_2 e_0 + A_3 e_1 + A_4 e_2 \\
                    &+ A_5 e_{01} + A_6 e_{20} + A_7 e_{12} + A_8 e_{012} \\
                    \underline{A} = \overline{A} = & \, A_8 + A_7 e_0 + A_6 e_1 + A_5 e_2 \\
                    &+ A_4 e_{01} + A_3 e_{20} + A_2 e_{12} + A_1 e_{012} \\
                \end{aligned}$$
        \section{Shapes and sizes}
            \par
                Let $ \mathbb{G}_{d,0,1} $ be a d-dimensional PGA:
                $$ \B{e_0}^2 = 0;\, \B{e_i}^2 = 1 $$
            \par
                Now we have a choice to make: \vspace{5px}
                \begin{enumerate}
                    \item \texttt{Point-based PGA}
                        \begin{enumerate}[-]
                            \item \texttt{vectors are points}
                            \item \texttt{(n-1)-vectors are hyperplanes}
                        \end{enumerate}
                    \item \texttt{Plane-based PGA}
                        \begin{enumerate}[-]
                            \item \texttt{vectors are hyperplanes}
                            \item \texttt{(n-1)-vectors are points}
                        \end{enumerate}
                \end{enumerate}
            \par
                Both of these are equally valid and many operations make use of computing in the dual algebra
                via $ \underline{\overline{A} \T{ op } \overline{B}} $.
                \begin{align*}
                    &point = e_0 + x e_1 + y e_2 + ... + w e_d \\
                    &hyperplane = \overline{e_0 + x e_1 + y e_2 + ... + w e_d}
                \end{align*}
                \boxedmath{\begin{aligned}
                    &hyperplane = e_0 + x e_1 + y e_2 + ... + w e_d \\
                    &point = \overline{e_0 + x e_1 + y e_2 + ... + w e_d}
                \end{aligned}}
                \aside{We will denote Plane-based operations inside boxes whenever they differ.}
            \par
                Let $ \braket{A}_k $ be the grade selection operator:
                \begin{gather*}
                    \braket{X_i}_k = \begin{cases}
                        X_i & (X_i \in \T{k-vector}) \\
                        0 & (X_i \notin \T{k-vector})
                    \end{cases} \\
                    \braket{A}_k = \sum_i \braket{X_i}_k
                \end{gather*}
            \par
                Then we can define the wedge $ \wedge $ and antiwedge $ \vee $ products:
                \begin{align*}
                    A \wedge B &= \sum_{j,k} \braket{\braket{A}_j \braket{B}_k}_{j+k} = \underline{\overline{A} \vee \overline{B}} \\ % TODO: check this
                    A \vee B &= \sum_{j,k} \braket{\braket{A}_j \braket{B}_k}_{n-((n-j)+(n-k))} \\
                    &= \underline{\overline{A} \wedge \overline{B}}
                \end{align*}
                \aside{These operators retain distributivity and associativity.}
            \par
                We can then join points into lines and lines into planes:
                \begin{align*}
                    line &= point_1 \join point_2 \\
                    plane &= point_1 \join point_2 \join point_3
                \end{align*} \vspace{-25px}
                \begin{align*}
                    A \join B = A \wedge B
                \end{align*}
                \boxedmath{\begin{aligned}
                    A \join B = A \vee B
                \end{aligned}}
                \aside{In fact this works with any two geometric objects, operators that don't have this property aren't really worth your time.}
            \par
                And we can meet two objects:
                \begin{align*}
                    &point = line_1 \meet line_2 \\
                    &point = plane \meet line \\
                    &line = plane_1 \meet plane_2
                \end{align*} \vspace{-25px}
                \begin{align*}
                    A \meet B = A \vee B
                \end{align*}
                \boxedmath{\begin{aligned}
                    A \meet B = A \wedge B
                \end{aligned}}
            \par
                If you meet two parallel lines, you get a point at infinity = an infinite point:
                \begin{align*}
                    line_1 \meet line_1 = x e_1 + y e_2 + ... + w e_d % TODO: check this
                \end{align*}
                \boxedmath{\begin{aligned}
                    line_1 \meet line_1 = \overline{x e_1 + y e_2 + ... + w e_d}
                \end{aligned}}
                \aside{If you flip the direction of one of the lines,
                you get an infinite point in the other direction}
            \par
                All objects in PGA have an direction $ A^D $ and position $ A^P $:
                \begin{gather*}
                    A = A^D + A^P \\
                    A^D = \sum_i A_i X_i \quad (e_0 \in X_i) \\
                    A^P = \sum_i A_i X_i \quad (e_0 \notin X_i)
                \end{gather*}
                \boxedmath{\begin{gathered}
                    A^D = \sum_i A_i X_i \quad (e_0 \notin X_i) \\
                    A^P = \sum_i A_i X_i \quad (e_0 \in X_i)
                \end{gathered}}
            \par
                In 2D and 3D:
                \begin{align*}
                    point = e_0 &+ A^P \\
                    line = A^D &+ A^P
                \end{align*}
                \boxedmath{point = \overline{e_0} + A^P}
            \par
                In 3D:
                \begin{align*}
                    plane = A^D &+ A_{15} e_{123} \\
                    = normal &+ distance
                \end{align*}
                \boxedmath{plane = A^D + A_2 e_0}
            \par
                We can take norms to measure the direction or the position:
                \begin{align*}
                    \norm{A}_D &= \sqrt{\sum_i \B{A^D_i}^2} \\
                    \norm{A}_P &= \sqrt{\sum_i \B{A^P_i}^2} \\
                    \norm{A} &= \sqrt{\sum_i |\B{A_i X_i}^2|} = \norm{A}_P \\
                \end{align*}
                \boxedmath{\begin{aligned}
                    \norm{A} &= \norm{A}_D \\
                \end{aligned}}
                $$ A^{-1} = \tilde{A} \F{1}{\norm{A}^2} $$
                \aside{A is finite if $ \norm{A}_D \ne 0 $, therefore infinite objects have no direction.}
            %\vfill\null\columnbreak
            \par
                We can use this to calculate lengths/areas/volumes/...:
                \begin{align*}
                    length(edge\_loop) &= \sum_i \norm{line_i}_P \\ % TODO: check this in both algebras
                    area(edge\_loop) &= \F{1}{2} \sum_i \norm{line_i}_D \\
                    line_i &= p_i \join p_{i+1} \\
                    area(triangle\_mesh) &= \F{1}{2} \sum_i \norm{plane_i}_P \\
                    volume(triangle\_mesh) &= \F{1}{3} \sum_i \norm{plane_i}_D \\
                    plane_i &= p_i \join p_{i+1} \join p_{i+2}
                \end{align*}
            \subsection{Rasterization}
            \par
                We could just take the $ x,y $ components of a d-dimensional point and get an orthographic projection.
            \par
                But we could also just make a line from the camera (at the origin) to a point and then intersect it with a plane:
                \begin{gather*}
                    p_{perspective} = (p_{camera} \join p_{vertex}) \meet (xy\_plane) \\
                    xy\_plane = p_1 \join p_2 \join p_3 \\
                    depth(p_{vertex}) = \norm{p_{vertex}}_P
                \end{gather*}
                TODO: project(A, B): $ P = (A \cdot B^{-1}) B $ ? % TODO: define dot
                TODO: project(point\_at\_origin, B)
                TODO: project(plane\_at\_infinity, B)
            \subsection{Raytracing}
                \par
                    TODO: $ line_1 \wedge line_2 \\= signed\_distance(line_1, line_2) \norm{line_1^D \wedge line_2^D} $ \\
                    TODO: correlate of signed distance between lines: $ line_1 \vee line_2 $ -> ray-triangle intersection -> raytraced game \\
                    TODO: distance of line to A -> raytraced graphing calculator \\
                    TODO: ?
                    $$ distance(A, B) = \begin{cases}
                        \underline{A \wedge \overline{B}} & (type(A) = type(B)) \\
                        distance(A, project(A, B)) & (type(A) \ne type(B)) \\
                    \end{cases} $$
                    TODO: cos(A, B) = A dot A / (A.norm() B.norm()); sin(A, B) = ?
            \subsection{Raymarching}
                    TODO: signed distance of point to A -> raymarched graphing calculator \\
            \subsection{}
        \section{Motors}
            \par
                Taking $ e^{bivector} $ gives us a motor (motion operator), we can apply motors via $ M A \tilde{M} $:
                $$\begin{aligned}
                    motor &= e^{bivector} \\
                    rotor &= e^{bivector^D}; rotor &\in motor \\
                    translator &= e^{bivector^P}; translator &\in motor
                \end{aligned}$$
                \aside{In 3D, rotors are quaternions.} % TODO: and motors are dual quaternions?
            \par
                Where bivector blades are rotations. For example $ \F{\theta}{2} e_{12} $ is an xy rotation by $ \theta $ degrees
                and $ \F{d}{2} e_{01} $ is a translation by $ d $:
                \begin{gather*}
                    e^{\F{\theta}{2} e_{12}} = cos \,\F{\theta}{2} + \B{sin \,\F{\theta}{2}} e_{12} \\
                    e^{\F{d}{2} e_{01}} = 1 + \F{d}{2} e_{01} \\
                \end{gather*}
            \par
                We can interpolate motors with nlerp or slerp, where $ BA^{-1} $ is a transformation that brings A to B:
                \begin{gather*}
                    nlerp(t, A, B) = \F{lerp(t, A, B)}{\norm{lerp(t, A, B)}} \\
                    lerp(t, A, B) = (1-t) A + t B \\
                    slerp(t, A, B) = (BA^{-1})^t A
                \end{gather*}
                \aside{nlerp does not allow for unnormalized motors, but we will be normalizing them in the physics simulation anyways.}
            \par
                TODO: line forces per https://enki.ws/ganja.js/examples/coffeeshop.html\#sUwbwu9vR and https://bivector.net/PGADYN.html
        \section{Bonus}
            \par
                Useless operators that are just here for completeness:
                \begin{gather*}
                    X_i^\dagger = (-1)^\floor{k} X_i \texttt{ // involute} \\
                    \bar{X_i} = (-1)^\floor{k+k/2} X_i \texttt{ // conjugate}
                \end{gather*}
                $$ \norm{A}_\infty = \norm{A}_D $$
                \boxedmath{\begin{gathered}
                    \norm{A}_\infty = \norm{A}_P
                \end{gathered}}
    \end{multicols*}
\end{document}
