\documentclass[a4paper,14pt,twoside]{book}
\usepackage[english]{babel}
\usepackage[utf8]{inputenc}
\usepackage{soul}
\usepackage[colorlinks=true,linkcolor=red]{hyperref}
\usepackage{extsizes}
\usepackage{multicol}
\usepackage{amsmath}
\usepackage{amssymb}
\usepackage{mathtools}
\begin{document}
\newcommand*{\REF}[1]{\hyperref[{#1}]{\autoref*{#1} \nameref*{#1}}}
\newcommand{\LIM}[2]{\lim_{#1 \to #2}}
\newcommand{\D}{\F{d}{dx}}
\newcommand{\DD}[2]{\F{d #1}{d #2}}
\newcommand{\B}[1]{\left(#1\right)}
\newcommand{\F}[2]{\frac{#1}{#2}}
\newcommand{\T}[1]{\textrm{#1}}
\newcommand{\N}[1]{{\fontfamily{serif}\fontsize{10}{1cm}\selectfont \rlap{#1}}}
\newcommand*{\QED}{\hfill\ensuremath{\blacksquare}}

\title{Maths}
\author{Patrolin}

{\fontsize{14.4}{1cm}\selectfont
\maketitle
\tableofcontents
\chapter{Derivatives}
The derivative is the rate of change of a function at a given point$:$ \\
\begin{align*}
	\T{When } \overbrace{\D f(x)}^{\N{the derivative of f(x)}} & > 0 \T{, } f(x) \T{ is increasing.} \\
	\T{When } \D f(x)                                          & < 0 \T{, } f(x) \T{ is decreasing.} \\
	\T{When } \D f(x)                                          & = 0 \T{, } f(x) \T{ is constant.}
\end{align*}
\pagebreak
\section{Notation}
If x is a variable, then $dx$ is a small change in $x$. \\
$dx$ is close to $0$, but not equal to $0$. $0$ is $\F{1}{\infty}$, and $\infty$ is evil! \\
\begin{align*}
	&dx \to 0 \\
\end{align*}
If y is a function of $x$, then $dy$ and $df(x)$ are the change caused by $dx$. \\
\begin{align*}
	    y &= f(x) \\
	   dy &= df(x) \\
	df(x) &= \underbrace{f(x) - f(x-dx)}_{\N{left subderivative}} = \underbrace{f(x+dx) - f(x)}_{\N{right subderivative}} \\
\end{align*}
If the function is discontinuous, then the derivative is undefined. \\
\begin{align*}
	\D f(x) &= \F{df(x)}{dx} \\
\end{align*}
Putting all of these ideas together gives us the equation for a derivative$:$ \\
$$\begin{rcases*}
	f(x)' = \D f(x) = \DD{f(x)}{x} \\
	df(x) = f(x+dx) - f(x) \\
	dx \to 0 \  (dx \ne 0) \\
\end{rcases*} \D f(x) = \LIM{h}{0} \F{f(x+h)-f(x)}{h}$$
\section{Elementary derivatives}
\subsection{$\D a = 0$}
\begin{align*}
	&a \in R \\
	f(x)    &= a                               \\
	\D f(x) &= \LIM{h}{0} \F{f(x+h) - f(x)}{h} \\
	\D a    &= \LIM{h}{0} \F{a-a}{h}           \\
	\D a    &= \LIM{h}{0} \F{0}{h}             \\
	\D a    &= 0 \  \because \  h \ne 0
\end{align*}
\subsection{$\D x^a = ax^{a-1}$}
\begin{align*}
	d(x^a)        &= (x+dx)^a - x^a \\
	d(x^a)        &= x^a
	+ \  \underbrace{\begin{matrix}
			+ (dx + x + x + x + \ldots) \\
			+ (x + dx + x + x + \ldots) \\
			+ (x + x + dx + x + \ldots) \\
			+ (x + x + x + dx + \ldots) \\
			\vdots
		\end{matrix}}_{x^{a-1}dx}
	\left. \phantom{\begin{matrix} \\ \\ \\ \\ \\ \end{matrix}} \right\}
	\T{a times}
	+ (\ldots)dx^2
	- x^a \\
	d(x^a)        &= ax^{a-1}dx + (\ldots)dx^2 \\
	\DD{(x^a)}{x} &= ax^{a-1} + (\ldots)dx \  \because \  dx \ne 0 \\
	\D x^a        &= ax^{a-1} \  \because \  dx \to 0 \\
	& \qquad \Downarrow \\
	& \D x = 1 \\
	& \D x^2 = 2x \\
	& \D x^3 = 3x^2 \\
	& \qquad \  \  \vdots \\
\end{align*}
\subsection{$\D e^x = e^x$}
This is how the value of $e$ is defined. \\
$e = 2.718\ldots$
\subsection{$\D \ln{x} = \F{1}{x}$}
If you have a function of one variable, then take the change in that
function with respect to that variable (implicit differentiation). \\
\begin{align*}
	y           &= \ln{x} \\
	e^y         &= x \\
	d(e^y)      &= d(x) \\
	e^y * dy    &= 1*dx \\
	\F{dy}{dx}  &= \F{1}{e^y} \\
	\F{d}{dx} y &= \F{1}{x} \\
	\D \ln{x}   &= \F{1}{x} \\
\end{align*}
If you have a function of many variables, apply \REF{sec:Rules} to
break it down into functions of one variable. \\
\pagebreak
\subsection{Trigonometric functions}
\N{see \url{https://lbry.tv/@3Blue1Brown:b/derivative-formulas-through-geometry}}
\begin{align*}
	&\D \sin{x} = \cos{x}  \\
	&\D \cos{x} = -\sin{x}
\end{align*}
Plus \href{https://www.priklady.eu/cs/matematika/derivace/derivace-funkce.alej}{others} which you probably don't need. \\
\section{Rules} \label{sec:Rules}
\subsection{The derivative is a linear function}
The derivative of a sum is the sum of derivatives. \\
\begin{align*}
	&\D (f(x) + g(x)) = \D f(x) \  + \  \D g(x) \\
\end{align*}
Constants can be multiplied outside. \\
\begin{align*}
	&a \in R \\
	&\D (a*x) = a * \D x \\
\end{align*}
Proof is left as an excercise to the reader. \\
\pagebreak
\subsection{Product rule}
\N{see \url{https://lbry.tv/@3Blue1Brown:b/visualizing-the-chain-rule-and-product}}
TODO(): put in a picture instead of a link
\begin{align*}
	d(f(x)*g(x))         &= df(x)*g(x) + f(x)*dg(x) + df(x)*dg(x) \\
	\F{d(f(x)*g(x))}{dx} &= \F{df(x)}{dx}*g(x) + f(x)*\F{dg(x)}{dx} + \F{df(x)}{dx}*dg(x) \\
	\D (f(x)*g(x))       &= \D f(x) * g(x) + f(x) * \D g(x) \  \because \  dg(x) \to 0 \\
\end{align*}
\begin{align*}
	&\D (x \ln x) = 1*\ln x + \F{x}{x} = \ln x + 1 \\
	&\D (\sin{x} * x^2) = \cos{x} * x^2 + \sin{x} * 2x \\
	&\D \F{\sin{x}}{x} = \D (\sin{x} * 1/x) = \F{\cos{x}}{x} - \F{\sin{x}}{x^2} \\
\end{align*}
\pagebreak
\subsection{$\DD{}{g(x)} f(g(x))$}
Derivative of $f(g(x))$ with respect to $g(x)$. \\
You simply treat $g(x)$ as if it was $x$. \\
\begin{align*}
	&\DD{}{(x^2)} (\sin{x^2}) = \DD{}{y} (\sin{y}) = \cos{y} = \cos{x^2} \\
	&\DD{}{(\sin{x})} (\sin{x})^3 = 3(\sin{x})^2 \\
	&\DD{}{(\ln{x})} (\ln{x})^{-1} = -1(\ln{x})^{-2} \\
\end{align*}
\subsection{Chain rule}
\begin{align*}
	\D f(g(x)) &= \F{df(g(x))}{dx} = \F{df(g(x))}{g(x)} \  * \  \F{dg(x)}{dx} \\
	\D f(g(x)) &= \DD{}{g(x)} f(g(x)) \  * \  \DD{}{x} g(x) \\
\end{align*}
\begin{align*}
	&\D (\sin{x^2}) = \cos{x^2} * 2x \\
	&\D \F{1}{\sin{x}} = \D (\sin x)^{-1} = -1*(\sin{x})^{-2} \  * \  \cos{x} = -\F{\cos{x}}{(\sin{x})^2} \\
	&\D \ln e^x = \F{1}{e^x} * e^x = 1
\end{align*}
\newpage
This also leads to a formula for division. \\
\begin{align*}
	\D \F{1}{g(x)}    &= -1*g(x)^{-2} * \D g(x) = -\F{\D g(x)}{g(x)^2} \\
	\D \F{f(x)}{g(x)} &= \F{\D f(x)}{g(x)} + f(x) * \D g(x) \\
	\D \F{f(x)}{g(x)} &= \F{\D f(x)}{g(x)} - f(x)*\F{\D g(x)}{g(x)^2} \\
	\D \F{f(x)}{g(x)} &= \F{\D f(x) * g(x) \  - \  f(x) * \D g(x)}{g(x)^2} \\
\end{align*}
\subsection{L'Hopital's rule}
Relies on geometric proof!
\begin{align*}
	&\LIM{x}{a} \F{g(x)}{h(x)} = \LIM{x}{a} \F{dg(x)}{dh(x)}
\end{align*}
\begin{align*}
	\LIM{x}{0} \F{\sin{x}}{x} &= \LIM{x}{0} \F{\cos{x * dx}}{1 * dx} \\
	\LIM{x}{0} \F{\sin{x}}{x} &= \LIM{x}{0} \F{\cos{x}}{1} \  \because \  dx \ne 0 \\
	\LIM{x}{0} \F{\sin{x}}{x} &= 1
\end{align*}
\subsection{Derivative of inverse function}
With $g(x)$ being the inverse of $f(x):$
\begin{align*}
	g(f(x))           &= x \\
	\D f(x) * \D g(x) &= 1 \\
	\D g(x)           &= \F{1}{\D f(x)} \\
\end{align*}
\section{Composite derivatives}
\subsection{$\D e^{-x} = -e^{-x}$}
\begin{align*}
	\D e^{-x} &= e^{-x} * (-1) \\
	\D e^{-x} &= -e^{-x} \\
\end{align*}
\subsection{$\D a^x = a^x \ln a$}
Note that $\ln a^x = x \ln a$.
\begin{align*}
	a^x    &= e^{\ln a^x} \\
	a^x    &= e^{x \ln a} \\
	\D a^x &= e^{x \ln a} * \ln a \\
	\D a^x &= a^x \ln a \\
\end{align*}
\subsection{$\D \log_a{x} = \F{1}{x \ln a}$}
\begin{align*}
	y              &= \log_a{x} \\
	a^y            &= x \\
	d(a^y)         &= d(x) \\
	a^y \ln a * dy &= 1*dx \\
	\DD{y}{x}      &= \F{1}{a^y \ln a} \\
	\D y           &= \F{1}{a^{\log_a{x}} \ln a} \\
	\D \log_a{x}   &= \F{1}{x \ln a} \\
\end{align*}
\subsection{$\D x^x = x^x * (\ln x + 1)$}
\begin{align*}
	f(x)     &= x^x \\
	\ln f(x) &= \ln (x^x) \\
	\ln f(x) &= x \ln x \\
	f(x)     &= e^{x \ln x} \\
	\D x^x   &= e^{x \ln x} * (\ln x + 1) \\
	\D x^x   &= x^x * (\ln x + 1) \\
\end{align*}
\end{document}
