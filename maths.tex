\documentclass[a4paper,14pt,twoside]{book}
\usepackage[english]{babel}
\usepackage[utf8]{inputenc}
\usepackage{soul}
\usepackage[colorlinks=true,linkcolor=red]{hyperref}
\usepackage{extsizes}
\usepackage{multicol}
\usepackage{amsmath}
\usepackage{amssymb}
\usepackage{mathtools}
\begin{document}
\newcommand*{\REF}[1]{\hyperref[{#1}]{\autoref*{#1} \nameref*{#1}}}
\newcommand{\LIM}[2]{\lim_{#1 \to #2}}
\newcommand{\D}{\F{d}{dx}}
\newcommand{\DD}[2]{\F{d #1}{d #2}}
\newcommand{\B}[1]{\left(#1\right)}
\newcommand{\F}[2]{\frac{#1}{#2}}
\newcommand{\T}[1]{\text{#1}}
\newcommand{\UB}[2]{\underbrace{#1}_{\N{#2}}}
\newcommand{\OB}[2]{\overbrace{#1}^{\N{#2}}}
\newcommand{\N}[1]{{\fontsize{10}{1cm}\selectfont \rlap{#1}}}
\newcommand*{\QED}{\hfill\ensuremath{\blacksquare}}

\title{Maths}
\author{Patrolin}

{\fontsize{14.4}{1cm}\selectfont
\maketitle
\tableofcontents
\chapter{Derivatives}
The derivative is the rate of change at a specific point$:$
\begin{align*}
	\T{When } \D f(x) & > 0 \T{, } f(x) \T{ is increasing} \\
	\T{When } \D f(x) & < 0 \T{, } f(x) \T{ is decreasing} \\
	\T{When } \D f(x) & = 0 \T{, } f(x) \T{ is constant}
\end{align*}
\pagebreak
\section{Notation}
If x is a variable, then $dx$ is a small change in $x$. \\
$dx$ approaches $0$, but is not equal to $0$. \\
Instead we ask what happens as $dx$ gets closer and closer to $0:$ \\
\begin{align*}
	dx & \to 0 \\
\end{align*}
If y is a function of $x$, then $dy$ and $df(x)$ are the change caused by $dx:$
\begin{align*}
	y     & = f(x)                 \\
	      & \Rightarrow dy = df(x) \\
	df(x) & = f(x+dx) - f(x)       \\
\end{align*}
The derivative of $f(x)$ is the ratio between $df(x)$ and $dx:$
\begin{align*}
	\D f(x) & = \F{df(x)}{dx}
\end{align*}
Putting all of these ideas together gives us the equation for a derivative$:$
\begin{align*}
	\begin{rcases*}
		f(x)' = \D f(x) = \DD{f(x)}{x} \\
		df(x) = f(x+dx) - f(x) \\
		dx \to 0\ (dx \ne 0)
	\end{rcases*}
	\D f(x) = \LIM{h}{0} \F{f(x+h)-f(x)}{h}
\end{align*}
If we know the derivative, we can also compute $df(x):$
\begin{align*}
	\D f(x) & = \F{df(x)}{dx} \\
	df(x)   & = \D f(x) * dx
\end{align*}
\section{Elementary derivatives}
\subsection{By $\D f(x):$}
\begin{align*}
	a       & \in R                             \\
	f(x)    & = a                               \\
	\D f(x) & = \LIM{h}{0} \F{f(x+h) - f(x)}{h} \\
	\D a    & = \LIM{h}{0} \F{a-a}{h}           \\
	\D a    & = \LIM{h}{0} \F{0}{h}             \\
	\D a    & = 0\ \because\ h \ne 0
\end{align*}
\subsection{By $df(x):$}
\begin{align*}
	d(x^a)        & = (x+dx)^a - x^a               \\
	              & = x^a
	+ \overbrace{\begin{matrix}
			+ (dx + x + x + x + \ldots) \\
			+ (x + dx + x + x + \ldots) \\
			+ (x + x + dx + x + \ldots) \\
			+ (x + x + x + dx + \ldots) \\
			\vdots
		\end{matrix}}^{x^{a-1}dx}
	\left. \phantom{\begin{matrix} \\ \\ \\ \\ \\ \end{matrix}} \right\}
	\T{a times}
	+ (\ldots)dx^2
	- x^a                                          \\
	d(x^a)        & = ax^{a-1}dx + (\ldots)dx^2    \\
	\DD{(x^a)}{x} & = ax^{a-1} + (\ldots)dx        \\
	\D x^a        & = ax^{a-1}\ \because\ dx \to 0 \\
	              & \Rightarrow \D x = 1
\end{align*}
\subsection{By definition $:$}
\begin{align*}
	\D e^x = e^x
\end{align*}
\subsection{By implicit differentiation $:$}
This involves taking the derivatives of each variable separately.
The variables may also be on one side of the equation.
\begin{align*}
	y           & = \ln{x}     \\
	e^y         & = x          \\
	d(e^y)      & = d(x)       \\
	e^y dy      & = 1dx        \\
	\F{dy}{dx}  & = \F{1}{e^y} \\
	\F{d}{dx} y & = \F{1}{x}   \\
	\D \ln{x}   & = \F{1}{x}
\end{align*}
\subsection{By fancy geometry $:$}
\N{see \url{https://lbry.tv/@3Blue1Brown:b/derivative-formulas-through-geometry}}
\begin{align*}
	\D \sin{x} & = \cos{x}  \\
	\D \cos{x} & = -\sin{x}
\end{align*}
Plus other \href{https://www.priklady.eu/cs/matematika/derivace/derivace-funkce.alej}{trigonometric functions} which you probably don't need.
\subsection{By giving up $:$}
\begin{align*}
	\D x^x       & = x^x(\ln{x} + 1) \\
	\D e^{-x}    & = -e^{-x}         \\
	\D a^x       & = a^x*\ln{a}      \\
	\D \log_a{x} & = \F{1}{x\ln{a}}
\end{align*}
\pagebreak
\section{Rules}
\subsection{Sum rule}
\N{see \url{https://lbry.tv/@3Blue1Brown:b/derivative-formulas-through-geometry}}
\begin{align*}
	\D (g(x) + h(x)) & = \D g(x) + \D h(x)
\end{align*}
\begin{align*}
	\D (\sin{x} + x^2) & = \D \sin{x} + \D x^2 = \cos{x} + 2x \\
	\D (x + 1)         & = 1 + 0 = 1                          \\
	\D (x^3 + e^x)     & = 3x^2 + e^x
\end{align*}
\pagebreak
\subsection{Product rule}
\N{see \url{https://lbry.tv/@3Blue1Brown:b/visualizing-the-chain-rule-and-product}}
\begin{align*}
	d(g(x) * h(x))        & = dg(x)*h(x) + g(x)*dh(x) + dg(x)*dh(x)                           \\
	\DD{(g(x) * h(x))}{x} & = \DD{g(x)}{x}*h(x) + g(x)*\DD{h(x)}{x} + \DD{g(x)}{x}*h(x)*dh(x) \\
	\D (g(x) * h(x))      & = \D g(x) * h(x) + g(x) * \D h(x)\ \because\ dh(x) \to 0
\end{align*}
\begin{align*}
	\D (\sin{x} * x^2)
	 & = \D \sin{x} * x^2 + \sin{x} * \D x^2
	= \cos{x} * x^2 + \sin{x} * 2x           \\
	\D (2 * x)
	 & = 0 * x + 2 * 1
	= 2                                      \\
	\D \F{\sin{x}}{x}
	 & = \D (\sin{x} * 1/x)
	= \F{\cos{x}}{x} - \F{\sin{x}}{x^2}
\end{align*}
This also leads to a formula for division $:$
\begin{align*}
	\D \F{g(x)}{h(x)} & = \F{\D g(x) * h(x) - g(x) * \D h(x)}{h(x)^2}
\end{align*}
\pagebreak
\subsection{Derivative with respect to $f(x):$} \label{d/df(x)}
You simply treat $f(x)$ as if it was $x:$
\begin{align*}
	\DD{}{(x^2)} \sin{(x^2)}
	= \cos{(x^2)}  \\
	\DD{}{(\sin{x})} (\sin{x})^3
	= 3(\sin{x})^2 \\
\end{align*}
\subsection{Chain rule}
\begin{align*}
	\D g(h(x)) & = \DD{g(h(x))}{x}                     \\
	\D g(h(x)) & = \DD{g(h(x))}{h(x)} * \DD{h(x)}{x}   \\
	\D g(h(x)) & = \DD{}{h(x)} g(h(x)) * \DD{}{x} h(x) \\
\end{align*}
\begin{align*}
	\D \sin{(x^2)}
	 & = \cos{(x^2)} * 2x              \\
	\D \F{1}{(\sin{x})}
	 & = \F{-1}{(\sin{x})^2} * \cos{x}
	= -\F{\cos{x}}{(\sin{x})^2}        \\
\end{align*}
\pagebreak
\subsection{L'Hopital's rule}
\begin{align*}
	\LIM{x}{a} \F{g(x)}{h(x)}
	 & = \LIM{x}{a} \F{dg(x)}{dh(x)}
\end{align*}
\begin{align*}
	\LIM{x}{0} \F{\sin{x}}{x} & = \LIM{x}{0} \F{\cos{x * dx}}{1 * dx}           \\
	\LIM{x}{0} \F{\sin{x}}{x} & = \LIM{x}{0} \F{\cos{x}}{1}\ \because\ dx \ne 0 \\
	\LIM{x}{0} \F{\sin{x}}{x} & = 1
\end{align*}
\subsection{Derivative of inverse function}
Given $g(x)$, the inverse of $f(x):$
\begin{align*}
	g(f(x))           & = x              \\
	\D f(x) * \D g(x) & = 1              \\
	\D g(x)           & = \F{1}{\D f(x)} \\
\end{align*}
\end{document}
