\documentclass[twoside]{article}
\usepackage{multicol}
\setlength{\columnsep}{4cm}
\setlength{\columnseprule}{0pt}
\usepackage[shortlabels]{enumitem}
\setlist[enumerate]{topsep=0pt,itemsep=-0.5ex,partopsep=0.5ex,parsep=0.5ex}

% fonts
    \usepackage[sc]{mathpazo} % Use the Palatino font
    \usepackage[T1]{fontenc} % Use 8-bit encoding that has 256 glyphs
    \linespread{1.05} % Line spacing - Palatino needs more space between lines
    \usepackage{microtype} % Slightly tweak font spacing for aesthetics

    \usepackage[english]{babel} % Language hyphenation and typographical rules

    \usepackage[hmarginratio=1:1,top=32mm,columnsep=20pt]{geometry} % Document margins
    \usepackage[hang, small,labelfont=bf,up,textfont=it,up]{caption} % Custom captions under/above floats in tables or figures
    \usepackage{booktabs} % Horizontal rules in tables

    \usepackage{titling} % Customizing the title section
    \usepackage{abstract} % Allows abstract customization
    \renewcommand{\abstractnamefont}{\normalfont\bfseries} % set abstract title to bold
    \renewcommand{\abstracttextfont}{\normalfont\small\itshape} % set abstract text to small italic
    \renewcommand{\maketitlehookd}{
        \begin{abstract}
        \abstracttext
        \end{abstract}
    }

    \usepackage{titlesec} % Allows customization of titles
    \setlength{\droptitle}{-4\baselineskip} % Move the title up
    \renewcommand\thesection{\Roman{section}} % Roman numerals for the sections
    \renewcommand\thesubsection{\roman{subsection}} % roman numerals for subsections
    \titleformat{\section}[block]{\large\scshape\centering}{\thesection.}{1em}{} % Change the look of the section titles
    \titleformat{\subsection}[block]{\large}{\thesubsection.}{1em}{} % Change the look of the section titles
    \pretitle{\begin{center}\Huge\bfseries} % Article title formatting
    \posttitle{\end{center}} % Article title closing formatting

    \usepackage{fancyhdr} % Headers and footers
    \pagestyle{fancy} % All pages have headers and footers
    \fancyhead{} % Blank out the default header
    \fancyfoot{} % Blank out the default footer
    \fancyhead[C]{} % Custom header text
    \fancyfoot[RO,LE]{\thepage} % Custom footer text

%\setlength{\skip\footins}{10mm}
%\setlength{\parskip}{1ex}

% utility functions
    \newcommand{\M}[1]{\scriptstyle{#1}} % math text
    \newcommand{\T}[1]{\textrm{#1}} % text in math mode
    \newcommand{\B}[1]{\left(#1\right)} % big brackets
    \newcommand{\F}[2]{\frac{#1}{#2}} % fraction
    \newcommand{\percent}{\T{ \% }}
    \newcommand{\pow}{\T{ }\M{**}\T{ }}
    \newcommand{\rem}{\T{ rem }}
    \newcommand{\mod}{\T{ mod }}
    \newcommand{\ifelse}[3]{#1\T{ ? }#2\T{ : }#3}
    \newcommand{\plusminus}{\pm}
    \newcommand{\minusplus}{\mp}
    \newcommand{\booland}{\T{ }\M{\&\&}\T{ }}
    \newcommand{\boolor}{\T{ }\M{||}\T{ }}
    \newcommand{\bitand}{\T{ }\M{\&}\T{ }}
    \newcommand{\bitor}{\T{ }\M{|}\T{ }}
    \newcommand{\bitxor}{\T{ }\raisebox{0.5ex}{$\scriptstyle\wedge$}\T{ }}
    \newcommand{\rightshift}{\T{ }\M{>>}\T{ }}
    \newcommand{\leftshift}{\T{ }\M{<<}\T{ }}
    \newcommand{\matrixmul}{\T{ }\M{@}\T{ }}

% header
\makeatletter % allow @ symbol as a character
\title{Representing mathematical expressions in plain text}
\author{\textsc{Patrolin}}
\date{\today}
\newcommand{\abstracttext}{
    Standard math notation often relies on spacing and/or 2D layout for legibility,
    this translates poorly to 1D (often monospace or nearly monospace) text used in programming and text files.
    Moreover the standard math notation is hard to explain in a few sentences and leads to confusion about operator precedence
    when you start adding many (domain-specific) operators ($\booland$, $\boolor$, $\leftshift$, $\rightshift$, $\bitand$, $\bitor$, $\bitxor$, $\matrixmul$).
}

% body
\begin{document}
    \maketitle
    \thispagestyle{empty}

    \begin{multicols*}{2}
        \section{Standard notation (SN)}
            Brackets are a good fallback, they let you express any expression you want although very verbosely.\vspace{1px}\\
            \centerline{$ (a + b) \cdot c $}
            \centerline{$ a + (b \cdot c) $}
            \centerline{$ a + (((b \cdot c) \cdot d) \cdot e) $}
            \par\vspace{5px}
                You probably don't want to write brackets all day,
                so SN has many alternative representations.
                \par
                E(MD)(AS) says that exponentiation $a^b$ takes precedence
                over multiplication $a\cdot{}b$ and division $\F{a}{b}$, both of which have the same precedence,
                and all of them take precedence over addition $a+b$ and subtraction $a-b$ (both of which have the same precedence).
                \par
                Therefore:\vspace{2px}\\
                \centerline{$ a + b \cdot c = a + (b \cdot c) $}
                \centerline{$ (a + b) \cdot c = (a + b) \cdot c $}\vspace{-1px}
            \par\vspace{5px}
                You may notice how inconsistent the presentation of these operators is,
                to write these in plain text we have to use someting like:\\
                $a\pow{}b$ for $a^b$, $a*b$ for $a\cdot{}b$, $a/b$ for $\F{a}{b}$
                \par
                You can also choose to write $a\cdot{}b$ as $ab$
                (or really $a\ b$ so we know where one name ends and the other begins),
                however this now has higher precedence than $a/b$, but still lower precedence than $a\pow{}b$.
            \par\vspace{5px}
                Not to mention that exponentiation is right-associative,
                meaning $a^{b^c} = a^{\B{b^c}}$, unlike the rest of the operators,
                since the left-associative version can be represented differently $(a^b)^c = a^{bc}$,
                however in practice nobody uses the right-associative version because
                \begin{enumerate}[a)]
                    \item it is unintuitive
                    \item nobody wants to or has to deal with numbers of this magnitude anyway.
                \end{enumerate}
            \centerline{}\vspace{5px}
                You may also encounter functions like $ f(a, b) = a \cdot b + b^2 $,
                then you can evaluate the function at some point $ f(3,5) = 3\cdot5 + 5^2 = 40 $
                \par
                If a function only has one variable you can write it without brackets
                (usually this is only done with $\sin$ and $\cos$):\vspace{2px}\\
                \centerline{$ \sin x = \sin(x) $}
                \par
                    This prefix operator then requires its own precedence; below $ab$, but above $a\cdot{}b$:\vspace{2px}\\
                \centerline{$ \sin xe^x = \sin(x \cdot (e \pow x)) $}
                \centerline{$ \sin x \cdot e^x = \sin(x) \cdot (e \pow x) $}
            \par\vspace{5px}
                This means that SN actually uses EIP(MD)(AS):
                \begin{enumerate}[1.]
                    \item Exponentiation
                    \item Implicit multiplication
                    \item Prefix operators
                    \item Multiplication/Division
                    \item Addition/Subtraction
                \end{enumerate}
        \section{Quick notation (QN)}
            Parsing long expressions can be complicated (especially for humans),
            so why not just make left-to-right the default.
            \begin{enumerate}[a)]
                \item every operator has the same precedence
                \item every operator is left-associative
            \end{enumerate}
            \par\vspace{2px}
                Therefore:\vspace{2px}\\
                \centerline{$ a + b * c = (a + b) * c $}
                \centerline{$ a + (b * c) = a + (b * c) $}\vspace{-1px}
            \par\vspace{5px}
                This makes adding new operators trivial
                and hopefully reduces the brackets needed per expression,
                like with polyfilling the modulo operator in Javascript:\vspace{4px}\\
                \centerline{$ a \T{ mod } b = a \percent b + b \percent b = ((a \percent b) + b) \percent b $}
            \par\vspace{5px}
                A consequence of this system is that
                you may have to write more brackets for operators with higher precedence in SN,
                however this really only applies to addition/subtraction followed by multiplication/division:\vspace{2px}\\
                \centerline{$ a + bx = a + (b*x) $}
                \centerline{$ a - \F{b}{x} = a - (b/x) $}
                \centerline{$ a * b^x = a * (b \pow x) $}
            \centerline{}
                You could also include implicit multiplication $a\ b$,
                but this doesn't actually save you any characters over $a*b$.
    \end{multicols*}
    \section{Results}
        \begin{center}
        \begin{tabular}{ll}
            \hline
            %\multicolumn{3}{c}{Multi-column}\\
            Lerp SN & \texttt{lerp(t, a, b) = (1 - t) * a + t * b}\\
            Lerp QN & \texttt{lerp(t, a, b) = 1 - t * a + (t * b)}\vspace{5px}\\
            Modulo SN & \texttt{a mod b = ((a rem b) + b) rem b}\\
            Modulo QN & \texttt{a mod b = a rem b + b rem b}\vspace{5px}\\
            Modulo2 SN & \texttt{a mod b = (a rem b) + (b * (a < 0))}\\
            Modulo2 QN & \texttt{a mod b = (a rem b) + (b * (a < 0))}\vspace{5px}\\
            Remainder SN & \texttt{a rem b = (a mod b) - (b * (a < 0))}\\
            Remainder QN & \texttt{a rem b = a mod b - (b * (a < 0))}\vspace{5px}\\
            Quadratic formula SN & \texttt{(-b + (b ** 2 - 4 * a * c) ** 0.5) / (2 * a)}\\
            Quadratic formula QN & \texttt{(-b + (b ** 2 - (4 * a * c)) ** 0.5) / (2 * a)}\vspace{5px}\\
            Modular exponentiation SN & \texttt{x[n] = (x[n-1] * (b[i] ?‌ x[n-1] :‌ x[0])) mod p}\\
            Modular exponentiation QN & \texttt{x[n] = x[n-1] * (b[i] ?‌ x[n-1] :‌ x[0]) mod p}\vspace{5px}\\
            Xorshift excerpt SN & \texttt{(x <‌< k) | (x >‌> (64 - k))}\\
            Xorshift excerpt QN & \texttt{x <‌< k | (x >‌> (64 - k))}\vspace{5px}\\
            Mersenne twister excerpt SN & \texttt{y | ((y >‌> u) \& d)}\\
            Mersenne twister excerpt QN & \texttt{y | (y >‌> u \& d)}\\
            \hline
        \end{tabular}
        \end{center}
        \par\vspace{5px}
            Notice how "Mersenne twister excerpt SN" has brackets for \texttt{(y >‌> u) \& d},
            this may or may not be required depending on the programming language used,
            as everyone makes up their own precedence rules.
            \par
                In conclusion QN provides an easy to understand, easy to extend, perhaps shorter notation
                for mathematical expressions.
\end{document}
