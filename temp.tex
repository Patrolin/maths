\documentclass[11pt,a4paper,draft]{article}
\usepackage{scalerel,stackengine}
\stackMath
\usepackage[utf8]{inputenc}
\usepackage{tikz} %for dash
\def\Overline#1{\ThisStyle{\overline{\addstackgap[.4pt]{\SavedStyle#1}}}}

\makeatletter

\newcommand{\dashover}[1]{%
  \ThisStyle{%
  \vbox {\m@th\ialign{##\crcr
  \preclosurefill \crcr
  \noalign{\kern-\p@\nointerlineskip}
  $\hfil\SavedStyle{#1}\hfil$\crcr}}}}

%% fill with (short) minus signs
\def\preclosurefill{%
  $\m@th%
  \xleaders\hbox{$\mkern0mu\shortbar\mkern0mu$}\hfill%
  \shortbar%
$}

%% make the minus shorter to fit \dashedleftarrow
\def\shortbar{%
  \smash{\scalebox{0.4}[1.0]{$-$}}}
\makeatother

\begin{document}

In order to distinguish between lattice operations $\bigvee, \bigwedge$ in structure A, we use the following symbols: $\Overline{\bigvee}, \Overline{\bigwedge}$. Similarly, to distinguish between operations $\bigvee, \bigwedge$ in structure B, we use the following symbols: $\dashover{\bigvee}, \dashover{\bigwedge}$.

Inline: $\Overline{\bigvee}_{\alpha \in I}$, $\dashover{\bigvee}_{\alpha \in I}$.

In separate line:
\[
\Overline{\bigvee_{\alpha \in I}}, \dashover{\bigvee_{\alpha \in I}}.
\]
\end{document}
